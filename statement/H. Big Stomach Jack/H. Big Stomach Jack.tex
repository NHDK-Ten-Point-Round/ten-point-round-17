\documentclass[12pt]{article}
\usepackage{xeCJK}
\usepackage{fontspec}
\usepackage[a4paper,top=2.8cm,bottom=2.8cm,left=2.3cm,right=2.3cm]{geometry}
\usepackage{graphicx}
\usepackage{listings}
\usepackage{xcolor}
\usepackage{indentfirst}
\usepackage{tikz}
\usepackage{amssymb}
\usepackage{amsthm}
\usepackage{amsmath}
\usepackage{fancyhdr}
\usepackage{tabularx}
\usepackage{hyperref}
\usepackage{ulem}
\usepackage{version}
\usepackage{thmtools}
\usepackage{qtree}
\usepackage{algpseudocode}
\usepackage{mathtools}
\usepackage{multicol}
\usepackage{enumitem}
\usepackage{ctable}
\usepackage{totcount}
\usepackage{textpos}

\XeTeXlinebreaklocale "zh"
\XeTeXlinebreakskip = 0pt plus 1pt

\setCJKmainfont[BoldFont={SourceHanSerifTW-SemiBold.otf},AutoFakeSlant]{SourceHanSerifTW-Regular.otf}
\setmonofont{JetBrainsMono-Regular.ttf}
\newfontfamily{\ProblemTitleMainFont}{SourceHanSerifTW-Bold.otf}
\newCJKfontfamily{\ProblemTitleCJKFont}{SourceHanSerifTW-Bold.otf}
\newcommand{\ProblemTitleFont}{\ProblemTitleMainFont\ProblemTitleCJKFont}

\pagestyle{fancy}

\lstset{
basicstyle=\footnotesize\ttfamily
}

% \raggedcolumns

\newcommand{\ProblemTitle}[2]{\noindent\Large{\ProblemTitleFont #1 (#2)}\normalsize\par}
\newcommand{\ProblemSection}[1]{\vspace{0.6cm}\par\noindent\large{\ProblemTitleFont #1}\normalsize\par}
\newcommand{\ProblemSubsection}[1]{\par\noindent{\ProblemTitleFont #1}\par}
\newcommand{\ProblemStatement}{\ProblemSection{問題敘述}}
\newcommand{\ProblemInput}{\ProblemSection{輸入說明}}
\newcommand{\ProblemOutput}{\ProblemSection{輸出說明}}
\newcommand{\ProblemConstraints}{\ProblemSection{測資限制}}

\newcommand{\ProblemSampleTitle}{\ProblemSection{範例測資}}

\newcounter{ProblemSample}
\newcommand{\ProblemSample}[2]{
    \stepcounter{ProblemSample}
    \noindent
    \begin{minipage}[t]{0.5\textwidth}
        \ProblemSubsection{範例輸入 \theProblemSample}
        \lstinputlisting{#1}
    \end{minipage}
    \begin{minipage}[t]{0.5\textwidth}
        \ProblemSubsection{範例輸出 \theProblemSample}
        \lstinputlisting{#2}
    \end{minipage}
}
\newenvironment{ProblemSampleWithNote}[2]{
    \stepcounter{ProblemSample}
    \noindent
    \begin{minipage}[t]{0.5\textwidth}
        \ProblemSubsection{範例輸入 \theProblemSample}
        \lstinputlisting{#1}
    \end{minipage}
    \begin{minipage}[t]{0.5\textwidth}
        \ProblemSubsection{範例輸出 \theProblemSample}
        \lstinputlisting{#2}
    \end{minipage}
    \vspace{-0.4cm}
    \ProblemSubsection{範例說明 \theProblemSample}
}{}

\newcommand{\ProblemSubtaskTitle}{\ProblemSection{評分說明}}
\newtotcounter{ProblemSubtask}
\newenvironment{ProblemSubtaskTable}{
    \begin{center}
        \begin{tabular}{ccl}
            \specialrule{1.3pt}{0pt}{1pt}
            子任務 & 分數 & 額外輸入限制 \\
            \specialrule{0.5pt}{1pt}{1pt}
}
{
            \specialrule{1.3pt}{1pt}{0pt}
        \end{tabular}
    \end{center}
}
\newcommand{\ProblemSubtask}[2]{ \stepcounter{ProblemSubtask} \theProblemSubtask & #1 & #2 \\ }

\setlist[enumerate]{itemsep=0pt, parsep=0pt, topsep=0pt}
\setlist[itemize]{itemsep=0pt, parsep=0pt, topsep=0pt}



\begin{document}


\begin{textblock}{0.1}(-0.6,-1.1)
	\includegraphics[height=1.2cm]{AA_YTP.png}
\end{textblock}


\renewcommand{\headrulewidth}{0pt}
\renewcommand{\baselinestretch}{1.3}
\pagenumbering{arabic}
\setlength\parindent{24pt}
\setlength\parskip{12pt}
\cfoot{\thepage}
\rhead{
	\small{NHDK Ten Point Round \#17 \\(Div.1+2, Sponsored by YTP)}
	
}

\ProblemTitle{H:Coding 夢之國的大胃王 Jack}{Big Stomach Jack}

\ProblemStatement

結完帳的 Jack 終於可以開始來吃烤肉了!雖然他的食量十分的恐怖,但是他並不是什麼食物都能吃很多的,和大多數的人一樣,只要是越好吃的東西,Jack 就越有動力吃更多。

已知有 $n$ 種食物,每種食物都有它的美味指數 $y_i$ 以及對 Jack 造成的飢餓指數 $h_i$,其中\textbf{飢餓指數的算法為 $\left \lfloor \sqrt{y_i} \right \rfloor$ (向下取整)}。

由於 Jack 是一名很有原則的人,他只要選定了餐桌上的一個起點,就會繼續往後吃,直到吃到了邊界或是不符合 Jack 的原則為止,而 Jack 吃東西的原則是:如果吃了這一道菜會無法符合條件,那麼 Jack 就不會吃下這道菜並停止進食。

餐廳老闆 Colten 為了讓菜餚的品質變高(\sout{其實是想讓 Jack 吃更多東西}),每次 Jack 吃完東西之後他都會在 Jack 所吃的區間 $[L,R]$ 選一個最難吃的食物,將其美味指數提升該次所吃的食物數量乘上一個常數 $k$ (例: 若 $k=2$,三道菜的美味指數 $y_1=10,\ y_2=11,\ y_3=13$,Jack 吃完之後會分別變成 16, 11, 13),若區間內有多個最難吃的食物,則會提升較前面的食物。

給定 $q$ 次詢問以及起點 $l$ ,請你找出以 $l$ 為起點開始吃的話 Jack 最多可以吃幾樣食材以及其美味指數的總和。



\ProblemInput

輸入第一行有三個正整數 $n,q,k$ 代表餐桌上有 $n$ 道食材、 $q$ 筆詢問以及常數 $k$

輸入第二行有 $n$ 個正整數 $y_1,y_2,...,y_n$ 代表每種食材的美味指數 

接著 $q$ 每行有一個正整數 $l_i$ 代表 Jack 開始吃的起點

\clearpage

\ProblemOutput

對於每次詢問輸出二正整數分別代表以 $l_i$ 為起點開始吃最多可以吃幾樣食材以及其美味指數的總和,並以空格間隔


\ProblemConstraints

\begin{itemize}
    \item $1 \le n,\ q \le 10^5$
    \item $k \le 100$
    \item $1 \le y_i \le 10^9$
    \item $1 \le l_i \le n$
\end{itemize}

\ProblemSampleTitle

%\begin{ProblemSampleWithNote}{ex1.in}{ex1.out}

%\end{ProblemSampleWithNote}

\ProblemSample{00_01.in}{00_01.out}

\ProblemSample{00_02.in}{00_02.out}

\clearpage

\ProblemSubtaskTitle

本題共有 \total{ProblemSubtask} 組子任務,條件限制如下所示。

\begin{ProblemSubtaskTable}
    \ProblemSubtask{11}{$1 \le n,\ q \le 1000,\ k=0$}
    \ProblemSubtask{17}{$1 \le n,\ q \le 1000$}
    \ProblemSubtask{29}{$k=0$}
    \ProblemSubtask{43}{無額外限制}
\end{ProblemSubtaskTable}

\end{document}