\documentclass[12pt]{article}
\usepackage{xeCJK}
\usepackage{fontspec}
\usepackage[a4paper,top=2.8cm,bottom=2.8cm,left=2.3cm,right=2.3cm]{geometry}
\usepackage{graphicx}
\usepackage{listings}
\usepackage{xcolor}
\usepackage{indentfirst}
\usepackage{tikz}
\usepackage{amssymb}
\usepackage{amsthm}
\usepackage{amsmath}
\usepackage{fancyhdr}
\usepackage{tabularx}
\usepackage{hyperref}
\usepackage{ulem}
\usepackage{version}
\usepackage{thmtools}
\usepackage{qtree}
\usepackage{algpseudocode}
\usepackage{mathtools}
\usepackage{multicol}
\usepackage{enumitem}
\usepackage{ctable}
\usepackage{totcount}
\usepackage{textpos}

\XeTeXlinebreaklocale "zh"
\XeTeXlinebreakskip = 0pt plus 1pt

\setCJKmainfont[BoldFont={SourceHanSerifTW-SemiBold.otf},AutoFakeSlant]{SourceHanSerifTW-Regular.otf}
\setmonofont{JetBrainsMono-Regular.ttf}
\newfontfamily{\ProblemTitleMainFont}{SourceHanSerifTW-Bold.otf}
\newCJKfontfamily{\ProblemTitleCJKFont}{SourceHanSerifTW-Bold.otf}
\newcommand{\ProblemTitleFont}{\ProblemTitleMainFont\ProblemTitleCJKFont}

\pagestyle{fancy}

\lstset{
basicstyle=\footnotesize\ttfamily
}

% \raggedcolumns

\newcommand{\ProblemTitle}[2]{\noindent\Large{\ProblemTitleFont #1 (#2)}\normalsize\par}
\newcommand{\ProblemSection}[1]{\vspace{0.6cm}\par\noindent\large{\ProblemTitleFont #1}\normalsize\par}
\newcommand{\ProblemSubsection}[1]{\par\noindent{\ProblemTitleFont #1}\par}
\newcommand{\ProblemStatement}{\ProblemSection{問題敘述}}
\newcommand{\ProblemInput}{\ProblemSection{輸入說明}}
\newcommand{\ProblemOutput}{\ProblemSection{輸出說明}}
\newcommand{\ProblemConstraints}{\ProblemSection{測資限制}}

\newcommand{\ProblemSampleTitle}{\ProblemSection{範例測資}}

\newcounter{ProblemSample}
\newcommand{\ProblemSample}[2]{
    \stepcounter{ProblemSample}
    \noindent
    \begin{minipage}[t]{0.5\textwidth}
        \ProblemSubsection{範例輸入 \theProblemSample}
        \lstinputlisting{#1}
    \end{minipage}
    \begin{minipage}[t]{0.5\textwidth}
        \ProblemSubsection{範例輸出 \theProblemSample}
        \lstinputlisting{#2}
    \end{minipage}
}
\newenvironment{ProblemSampleWithNote}[2]{
    \stepcounter{ProblemSample}
    \noindent
    \begin{minipage}[t]{0.5\textwidth}
        \ProblemSubsection{範例輸入 \theProblemSample}
        \lstinputlisting{#1}
    \end{minipage}
    \begin{minipage}[t]{0.5\textwidth}
        \ProblemSubsection{範例輸出 \theProblemSample}
        \lstinputlisting{#2}
    \end{minipage}
    \vspace{-0.4cm}
    \ProblemSubsection{範例說明 \theProblemSample}
}{}

\newcommand{\ProblemSubtaskTitle}{\ProblemSection{評分說明}}
\newtotcounter{ProblemSubtask}
\newenvironment{ProblemSubtaskTable}{
    \begin{center}
        \begin{tabular}{ccl}
            \specialrule{1.3pt}{0pt}{1pt}
            子任務 & 分數 & 額外輸入限制 \\
            \specialrule{0.5pt}{1pt}{1pt}
}
{
            \specialrule{1.3pt}{1pt}{0pt}
        \end{tabular}
    \end{center}
}
\newcommand{\ProblemSubtask}[2]{ \stepcounter{ProblemSubtask} \theProblemSubtask & #1 & #2 \\ }

\setlist[enumerate]{itemsep=0pt, parsep=0pt, topsep=0pt}
\setlist[itemize]{itemsep=0pt, parsep=0pt, topsep=0pt}



\begin{document}



\begin{textblock}{0.1}(-0.6,-1.1)
	\includegraphics[height=1.2cm]{AA_YTP.png}
\end{textblock}




\renewcommand{\headrulewidth}{0pt}
\renewcommand{\baselinestretch}{1.3}
\pagenumbering{arabic}
\setlength\parindent{24pt}
\setlength\parskip{12pt}
\cfoot{\thepage}
\rhead{
	\small{NHDK Ten Point Round \#17 \\(Div.1+2, Sponsored by YTP)}
	
}

\ProblemTitle{E. Koying 的彈跳床飛行記}{Flying Koying}

\ProblemStatement

Koying 今天來到了 Coding 夢之國遊玩,一聽到夢之國內有一個利用超大型彈跳床來讓遊客飛上天的設施,Koying 便迫不及待的前往報名,希望可以一圓他的飛行夢。

這個飛行設施建立在一個長度為 $n$ 的條狀場地上,其中在第 $1,...,n$ 的點上都配有一個超大型彈跳床。

每個彈跳床都有一個快樂指數 $h_i$ 代表使用那個彈跳床可以讓 Koying 獲得幾點的快樂,並且可以設定其要往前彈跳多遠,不過距離最多不能超過 $r$ 個單位。

舉例來說,如果 Koying 選擇在第 $1$ 個點的超大彈跳床往前跳兩個單位,且 $h_1 = 5$,那麼 Koying 就會在第 $3$ 個點落下並獲得 $5$ 點的快樂指數。

Koying 可以選擇要從哪個彈跳床開始遊玩,不過有以下的限制:Koying 如果跳出場地之外他就會摔死,所以絕對不可以超出場地,且 Koying 在落地之後一定得在落地的地方馬上再往前飛至少一單位,否則就要退出遊戲。請你告訴 Koying 在跳不超過 $k$ 次的情況下最多可以獲得幾點的快樂指數。

\ProblemInput

輸入第一行有四個正整數 $n,r,k$ 代表場地的長度、 彈跳床最多可以往前幾單位以及 Koying 最多可以飛行幾次

輸入第二行有 $n$ 個整數 $h_1,h_2,...,h_n$ 代表在每個點使用彈跳床所能獲得的快樂指數

\ProblemOutput

輸出一個整數代表 Koying 在不飛出場地以及不飛行超過 $k$ 次的情況下最多可以得到的快樂指數。

\clearpage

\ProblemConstraints

\begin{itemize}
    \item $1 \le n \le 5000$
    \item $1 \le r \le n$
    \item $1 \le k \le n$
    \item $1 \le h_i \le 10^5$
\end{itemize}

\ProblemSampleTitle

\begin{ProblemSampleWithNote}{00_01.in}{00_01.out}

在範例一中,最多只能飛一次,所以我們選擇在 $i=3$ 的點往前飛一單位到 $i=4$,並獲得 $8$ 點快樂指數

\end{ProblemSampleWithNote}

\begin{ProblemSampleWithNote}{00_02.in}{00_02.out}
在範例二中,雖然最佳解是從 $i=2$ 飛到 $i=4$ 再飛到 $i=5$,但因為彈跳床最多只能飛一單位,所以最佳情況是從 $i=2$ 飛到 $i=3$ 再飛到 $i=4$,獲得 $12$ 點快樂指數
\end{ProblemSampleWithNote}

%\ProblemSample{ex1.in}{ex1.out}

\ProblemSubtaskTitle

本題共有 \total{ProblemSubtask} 組子任務,條件限制如下所示。

\begin{ProblemSubtaskTable}
    \ProblemSubtask{15}{$r=n,\ n \le 500$} %O(nrk)前墜
    \ProblemSubtask{35}{$r=n$} % 前墜 DP   
    \ProblemSubtask{11}{$n \le 500$}  % O(nrk)
    \ProblemSubtask{39}{無額外限制} % O()
\end{ProblemSubtaskTable}

\end{document}