\documentclass[12pt]{article}
\usepackage{xeCJK}
\usepackage{fontspec}
\usepackage[a4paper,top=2.8cm,bottom=2.8cm,left=2.3cm,right=2.3cm]{geometry}
\usepackage{graphicx}
\usepackage{listings}
\usepackage{xcolor}
\usepackage{indentfirst}
\usepackage{tikz}
\usepackage{amssymb}
\usepackage{amsthm}
\usepackage{amsmath}
\usepackage{fancyhdr}
\usepackage{tabularx}
\usepackage{hyperref}
\usepackage{ulem}
\usepackage{version}
\usepackage{thmtools}
\usepackage{qtree}
\usepackage{algpseudocode}
\usepackage{mathtools}
\usepackage{multicol}
\usepackage{enumitem}
\usepackage{ctable}
\usepackage{totcount}
\usepackage{textpos}

\XeTeXlinebreaklocale "zh"
\XeTeXlinebreakskip = 0pt plus 1pt

\setCJKmainfont[BoldFont={SourceHanSerifTW-SemiBold.otf},AutoFakeSlant]{SourceHanSerifTW-Regular.otf}
\setmonofont{JetBrainsMono-Regular.ttf}
\newfontfamily{\ProblemTitleMainFont}{SourceHanSerifTW-Bold.otf}
\newCJKfontfamily{\ProblemTitleCJKFont}{SourceHanSerifTW-Bold.otf}
\newcommand{\ProblemTitleFont}{\ProblemTitleMainFont\ProblemTitleCJKFont}

\pagestyle{fancy}

\lstset{
basicstyle=\footnotesize\ttfamily
}

% \raggedcolumns

\newcommand{\ProblemTitle}[2]{\noindent\Large{\ProblemTitleFont #1 (#2)}\normalsize\par}
\newcommand{\ProblemSection}[1]{\vspace{0.6cm}\par\noindent\large{\ProblemTitleFont #1}\normalsize\par}
\newcommand{\ProblemSubsection}[1]{\par\noindent{\ProblemTitleFont #1}\par}
\newcommand{\ProblemStatement}{\ProblemSection{問題敘述}}
\newcommand{\ProblemInput}{\ProblemSection{輸入說明}}
\newcommand{\ProblemOutput}{\ProblemSection{輸出說明}}
\newcommand{\ProblemConstraints}{\ProblemSection{測資限制}}

\newcommand{\ProblemSampleTitle}{\ProblemSection{範例測資}}

\newcounter{ProblemSample}
\newcommand{\ProblemSample}[2]{
    \stepcounter{ProblemSample}
    \noindent
    \begin{minipage}[t]{0.5\textwidth}
        \ProblemSubsection{範例輸入 \theProblemSample}
        \lstinputlisting{#1}
    \end{minipage}
    \begin{minipage}[t]{0.5\textwidth}
        \ProblemSubsection{範例輸出 \theProblemSample}
        \lstinputlisting{#2}
    \end{minipage}
}
\newenvironment{ProblemSampleWithNote}[2]{
    \stepcounter{ProblemSample}
    \noindent
    \begin{minipage}[t]{0.5\textwidth}
        \ProblemSubsection{範例輸入 \theProblemSample}
        \lstinputlisting{#1}
    \end{minipage}
    \begin{minipage}[t]{0.5\textwidth}
        \ProblemSubsection{範例輸出 \theProblemSample}
        \lstinputlisting{#2}
    \end{minipage}
    \vspace{-0.4cm}
    \ProblemSubsection{範例說明 \theProblemSample}
}{}

\newcommand{\ProblemSubtaskTitle}{\ProblemSection{評分說明}}
\newtotcounter{ProblemSubtask}
\newenvironment{ProblemSubtaskTable}{
    \begin{center}
        \begin{tabular}{ccl}
            \specialrule{1.3pt}{0pt}{1pt}
            子任務 & 分數 & 額外輸入限制 \\
            \specialrule{0.5pt}{1pt}{1pt}
}
{
            \specialrule{1.3pt}{1pt}{0pt}
        \end{tabular}
    \end{center}
}
\newcommand{\ProblemSubtask}[2]{ \stepcounter{ProblemSubtask} \theProblemSubtask & #1 & #2 \\ }

\setlist[enumerate]{itemsep=0pt, parsep=0pt, topsep=0pt}
\setlist[itemize]{itemsep=0pt, parsep=0pt, topsep=0pt}



\begin{document}


%\begin{textblock}{3.8}(9.7,-1.2)
	%\includegraphics[height=1.4cm]{NHDK_B.png}
%\end{textblock}

\begin{textblock}{0.1}(-0.6,-1.1)
	\includegraphics[height=1.2cm]{AA_YTP.png}
\end{textblock}




\renewcommand{\headrulewidth}{0pt}
\renewcommand{\baselinestretch}{1.3}
\pagenumbering{arabic}
\setlength\parindent{24pt}
\setlength\parskip{12pt}
\cfoot{\thepage}
\rhead{
	\small{NHDK Ten Point Round \#17 \\(Div.1+2, Sponsored by YTP)}
	
}

\ProblemTitle{B. 豪邁大笑的Koying}{Laughing Koying}

\ProblemStatement

乾,你害我上課看著下面笑出來

如果是經常在上課時間跟 Koying 聊天的人一定對這句話不陌生,因為每當聊到有趣好笑的事情,笑點低下的 Koying 就會莫名地看著手機笑了出來,然後就會出現最開頭的那句話。

雖然 Koying 的笑點低下,但是每天 Koying 的笑點也會因為他上上下下的情緒而有起伏,因此每天的笑點也會不盡相同。

每天 Koying 都會接受到一件好笑指數為 $l_i$ 的事情,但只有當 $l_i \ge$ 好笑指數門檻 $k$ 時 Koying 才會把他記起來,且接收完該天的好笑事情之後,若 Koying 還記得的好笑事情的好笑指數總和 $\ge$ 當天的笑點 $a_i$,他就會豪邁大笑並且把所有的事情都忘光光。

給定好笑指數門檻 $k$ 、每天接收到的好笑指數 $l_i$ 以及該天的笑點 $a_i$,請幫忙算出 Koying 會豪邁大笑幾次以及會在哪幾天豪邁大笑吧!

\ProblemInput

第一行有二非負整數 $n,\ k$,代表總共有 $n$ 天以及好笑指數門檻 $k$ 代表每天的好笑指數需要 $\ge k$ 才會被 Koying 記起來。

第二行有 $n$ 個正整數代表 Koying 在第 $i$ 天的笑點 $a_i\ (1 \le i \le n)$。

第三行有 $n$ 個正整數代表 Koying 在時間第 $i$ 天所接收到的好笑指數 $l_i\ (1 \le i \le n)$。


\ProblemOutput

輸出第一行包含一非負整數 $cnt$ 代表 Koying 豪邁大笑了 $cnt$ 次。

第二行有 $cnt$ 個正整數代表由小到大代表 Koying 在哪幾天會豪邁大笑,若沒有豪邁大笑則不輸出第二行。

\clearpage

\ProblemConstraints

\begin{itemize}
	\item $1 \le n \le 10^5$
	\item $0 \le k \le 10^5$
    \item $1 \le a_i,\ l_i \le 10^9$
\end{itemize}

\ProblemSampleTitle

\ProblemSample{1in.txt}{1out.txt}

\begin{ProblemSampleWithNote}{2in.txt}{2out.txt}
第一天因為 Koying 得到的好笑指數 $< k$ ,所以不會被 Koying 記住。

而第二、四天所累加的好笑指數皆 $\ge k$ 該天的笑點,因此 Koying 會在這幾天豪邁大笑。
\end{ProblemSampleWithNote}

\ProblemSample{3in.txt}{3out.txt}



\ProblemSubtaskTitle

本題共有 \total{ProblemSubtask} 組子任務,條件限制如下所示。

\begin{ProblemSubtaskTable}
    \ProblemSubtask{30}{$k=0$}
    \ProblemSubtask{70}{無額外限制}
\end{ProblemSubtaskTable}

\end{document}