\documentclass[12pt]{article}
\usepackage{xeCJK}
\usepackage{fontspec}
\usepackage[a4paper,top=2.8cm,bottom=2.8cm,left=2.3cm,right=2.3cm]{geometry}
\usepackage{graphicx}
\usepackage{listings}
\usepackage{xcolor}
\usepackage{indentfirst}
\usepackage{tikz}
\usepackage{amssymb}
\usepackage{amsthm}
\usepackage{amsmath}
\usepackage{fancyhdr}
\usepackage{tabularx}
\usepackage{hyperref}
\usepackage{ulem}
\usepackage{version}
\usepackage{thmtools}
\usepackage{qtree}
\usepackage{algpseudocode}
\usepackage{mathtools}
\usepackage{multicol}
\usepackage{enumitem}
\usepackage{ctable}
\usepackage{totcount}
\usepackage{textpos}

\XeTeXlinebreaklocale "zh"
\XeTeXlinebreakskip = 0pt plus 1pt

\setCJKmainfont[BoldFont={SourceHanSerifTW-SemiBold.otf},AutoFakeSlant]{SourceHanSerifTW-Regular.otf}
\setmonofont{JetBrainsMono-Regular.ttf}
\newfontfamily{\ProblemTitleMainFont}{SourceHanSerifTW-Bold.otf}
\newCJKfontfamily{\ProblemTitleCJKFont}{SourceHanSerifTW-Bold.otf}
\newcommand{\ProblemTitleFont}{\ProblemTitleMainFont\ProblemTitleCJKFont}

\pagestyle{fancy}

\lstset{
basicstyle=\footnotesize\ttfamily
}

% \raggedcolumns

\newcommand{\ProblemTitle}[2]{\noindent\Large{\ProblemTitleFont #1 (#2)}\normalsize\par}
\newcommand{\ProblemSection}[1]{\vspace{0.6cm}\par\noindent\large{\ProblemTitleFont #1}\normalsize\par}
\newcommand{\ProblemSubsection}[1]{\par\noindent{\ProblemTitleFont #1}\par}
\newcommand{\ProblemStatement}{\ProblemSection{問題敘述}}
\newcommand{\ProblemInput}{\ProblemSection{輸入說明}}
\newcommand{\ProblemOutput}{\ProblemSection{輸出說明}}
\newcommand{\ProblemConstraints}{\ProblemSection{測資限制}}

\newcommand{\ProblemSampleTitle}{\ProblemSection{範例測資}}

\newcounter{ProblemSample}
\newcommand{\ProblemSample}[2]{
    \stepcounter{ProblemSample}
    \noindent
    \begin{minipage}[t]{0.5\textwidth}
        \ProblemSubsection{範例輸入 \theProblemSample}
        \lstinputlisting{#1}
    \end{minipage}
    \begin{minipage}[t]{0.5\textwidth}
        \ProblemSubsection{範例輸出 \theProblemSample}
        \lstinputlisting{#2}
    \end{minipage}
}
\newenvironment{ProblemSampleWithNote}[2]{
    \stepcounter{ProblemSample}
    \noindent
    \begin{minipage}[t]{0.5\textwidth}
        \ProblemSubsection{範例輸入 \theProblemSample}
        \lstinputlisting{#1}
    \end{minipage}
    \begin{minipage}[t]{0.5\textwidth}
        \ProblemSubsection{範例輸出 \theProblemSample}
        \lstinputlisting{#2}
    \end{minipage}
    \vspace{-0.4cm}
    \ProblemSubsection{範例說明 \theProblemSample}
}{}

\newcommand{\ProblemSubtaskTitle}{\ProblemSection{評分說明}}
\newtotcounter{ProblemSubtask}
\newenvironment{ProblemSubtaskTable}{
    \begin{center}
        \begin{tabular}{ccl}
            \specialrule{1.3pt}{0pt}{1pt}
            子任務 & 分數 & 額外輸入限制 \\
            \specialrule{0.5pt}{1pt}{1pt}
}
{
            \specialrule{1.3pt}{1pt}{0pt}
        \end{tabular}
    \end{center}
}
\newcommand{\ProblemSubtask}[2]{ \stepcounter{ProblemSubtask} \theProblemSubtask & #1 & #2 \\ }

\setlist[enumerate]{itemsep=0pt, parsep=0pt, topsep=0pt}
\setlist[itemize]{itemsep=0pt, parsep=0pt, topsep=0pt}



\begin{document}



\begin{textblock}{0.1}(-0.6,-1.1)
	\includegraphics[height=1.2cm]{AA_YTP.png}
\end{textblock}




\renewcommand{\headrulewidth}{0pt}
\renewcommand{\baselinestretch}{1.3}
\pagenumbering{arabic}
\setlength\parindent{24pt}
\setlength\parskip{12pt}
\cfoot{\thepage}
\rhead{
	\small{NHDK Ten Point Round \#17 \\(Div.1+2, Sponsored by YTP)}
	
}

\ProblemTitle{A. Colten 的撞球對決記}{Pool}

\ProblemStatement

今天 Colten 和 Koying 以及 HPK 一同到了台南的永華運動中心打撞球,但由於人數有三人,而撞球是兩邊對抗的運動,所以他們需要透過分隊來進行比賽

比賽由兩隊輪流上場,一開始由第一隊的人上場,接下來換第二隊,再輪回第一隊,若一隊有兩個人則兩人會輪流在輪到該隊時上場,以此類推

舉例來說,若第一隊的隊員有 Colten 以及 Koying,第二隊有 HPK,那麼上場的順序將是 Colten, HPK, Koying, HPK

今天給你兩隊的人數以及隊員,請你告訴記性不好的 Koying,在第一隊先攻的情況下,第 $q$ 次攻擊是輪到誰吧

\ProblemInput

輸入第一行有三個正整數 $n,m,t$ 代表兩隊的人數以及查詢次數

輸入第二行有 $n$ 個字串代表第一隊的隊員

輸入第三行有 $m$ 個字串代表第二隊的隊員

接下來的 $t$ 行每行有一個正整數 $q$ 代表 Koying 想要查詢第 $q$ 次是輪到誰上場

\ProblemOutput

輸出 $t$ 行,每行包含一個字串代表每次查詢的結果


\clearpage

\ProblemConstraints

\begin{itemize}
    \item $n + m \in \{2,3\}$
    \item $1 \le t \le 1000$
    \item $1 \le q \le 10^9$
    \item 隊員只會出現 Colten, Koying, HPK 三人
\end{itemize}

\ProblemSampleTitle

%\begin{ProblemSampleWithNote}{ex1.in}{ex1.out}

%\end{ProblemSampleWithNote}

\ProblemSample{1in.txt}{1out.txt}

\ProblemSample{2in.txt}{2out.txt}


\ProblemSubtaskTitle

本題共有 \total{ProblemSubtask} 組子任務,條件限制如下所示。

\begin{ProblemSubtaskTable}
    \ProblemSubtask{20}{$n+m=2$}
    \ProblemSubtask{80}{無額外限制}
\end{ProblemSubtaskTable}

\end{document}