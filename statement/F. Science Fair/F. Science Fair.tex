\documentclass[12pt]{article}
\usepackage{xeCJK}
\usepackage{fontspec}
\usepackage[a4paper,top=2.8cm,bottom=2.8cm,left=2.3cm,right=2.3cm]{geometry}
\usepackage{graphicx}
\usepackage{listings}
\usepackage{xcolor}
\usepackage{indentfirst}
\usepackage{tikz}
\usepackage{amssymb}
\usepackage{amsthm}
\usepackage{amsmath}
\usepackage{fancyhdr}
\usepackage{tabularx}
\usepackage{hyperref}
\usepackage{ulem}
\usepackage{version}
\usepackage{thmtools}
\usepackage{qtree}
\usepackage{algpseudocode}
\usepackage{mathtools}
\usepackage{multicol}
\usepackage{enumitem}
\usepackage{ctable}
\usepackage{totcount}
\usepackage{textpos}

\XeTeXlinebreaklocale "zh"
\XeTeXlinebreakskip = 0pt plus 1pt

\setCJKmainfont[BoldFont={SourceHanSerifTW-SemiBold.otf},AutoFakeSlant]{SourceHanSerifTW-Regular.otf}
\setmonofont{JetBrainsMono-Regular.ttf}
\newfontfamily{\ProblemTitleMainFont}{SourceHanSerifTW-Bold.otf}
\newCJKfontfamily{\ProblemTitleCJKFont}{SourceHanSerifTW-Bold.otf}
\newcommand{\ProblemTitleFont}{\ProblemTitleMainFont\ProblemTitleCJKFont}

\pagestyle{fancy}

\lstset{
basicstyle=\footnotesize\ttfamily
}

% \raggedcolumns

\newcommand{\ProblemTitle}[2]{\noindent\Large{\ProblemTitleFont #1 (#2)}\normalsize\par}
\newcommand{\ProblemSection}[1]{\vspace{0.6cm}\par\noindent\large{\ProblemTitleFont #1}\normalsize\par}
\newcommand{\ProblemSubsection}[1]{\par\noindent{\ProblemTitleFont #1}\par}
\newcommand{\ProblemStatement}{\ProblemSection{問題敘述}}
\newcommand{\ProblemInput}{\ProblemSection{輸入說明}}
\newcommand{\ProblemOutput}{\ProblemSection{輸出說明}}
\newcommand{\ProblemConstraints}{\ProblemSection{測資限制}}

\newcommand{\ProblemSampleTitle}{\ProblemSection{範例測資}}

\newcounter{ProblemSample}
\newcommand{\ProblemSample}[2]{
    \stepcounter{ProblemSample}
    \noindent
    \begin{minipage}[t]{0.5\textwidth}
        \ProblemSubsection{範例輸入 \theProblemSample}
        \lstinputlisting{#1}
    \end{minipage}
    \begin{minipage}[t]{0.5\textwidth}
        \ProblemSubsection{範例輸出 \theProblemSample}
        \lstinputlisting{#2}
    \end{minipage}
}
\newenvironment{ProblemSampleWithNote}[2]{
    \stepcounter{ProblemSample}
    \noindent
    \begin{minipage}[t]{0.5\textwidth}
        \ProblemSubsection{範例輸入 \theProblemSample}
        \lstinputlisting{#1}
    \end{minipage}
    \begin{minipage}[t]{0.5\textwidth}
        \ProblemSubsection{範例輸出 \theProblemSample}
        \lstinputlisting{#2}
    \end{minipage}
    \vspace{-0.4cm}
    \ProblemSubsection{範例說明 \theProblemSample}
}{}

\newcommand{\ProblemSubtaskTitle}{\ProblemSection{評分說明}}
\newtotcounter{ProblemSubtask}
\newenvironment{ProblemSubtaskTable}{
    \begin{center}
        \begin{tabular}{ccl}
            \specialrule{1.3pt}{0pt}{1pt}
            子任務 & 分數 & 額外輸入限制 \\
            \specialrule{0.5pt}{1pt}{1pt}
}
{
            \specialrule{1.3pt}{1pt}{0pt}
        \end{tabular}
    \end{center}
}
\newcommand{\ProblemSubtask}[2]{ \stepcounter{ProblemSubtask} \theProblemSubtask & #1 & #2 \\ }

\setlist[enumerate]{itemsep=0pt, parsep=0pt, topsep=0pt}
\setlist[itemize]{itemsep=0pt, parsep=0pt, topsep=0pt}



\begin{document}


%\begin{textblock}{3.8}(9.7,-1.2)
	%\includegraphics[height=1.4cm]{NHDK_B.png}
%\end{textblock}

\begin{textblock}{0.1}(-0.6,-1.1)
	\includegraphics[height=1.2cm]{AA_YTP.png}
\end{textblock}




\renewcommand{\headrulewidth}{0pt}
\renewcommand{\baselinestretch}{1.3}
\pagenumbering{arabic}
\setlength\parindent{24pt}
\setlength\parskip{12pt}
\cfoot{\thepage}
\rhead{
	\small{NHDK Ten Point Round \#17 \\(Div.1+2, Sponsored by YTP)}
	
}

\ProblemTitle{F. Colten 的科學展覽會}{Science Fair}

\ProblemStatement

Colten 最近正在準備即將到來的科學展覽會(以下簡稱科展),只不過隨著日子一天一天過去,科展的準備時間也越來越緊迫,他們需要一個有效率(\sout{偷爛})的方法來比對兩段聲音,在黑科技的幫助下,他們成功地將聲音轉換成了字串,不過由於黑科技尚未完善,所以在比較上仍然需要 Colten 的團隊自行處理


聰明的 Colten 很快就找到了利用\textbf{編輯距離}來比對兩筆資料的相似程度這個方法,若兩個字串 $s_1,s_2$ 之間的編輯距離為 $d$ ,兩個字串的平均長度為 $a$,則相似程度的算法為 $(1-\frac{d}{a}) \times 100\%$

至於該如何判定兩筆資料是否相同呢?Colten 想到了一個方法:Colten 會在兩筆資料中各取一個長度為 $x$ 且 $x \ge 10$ 的子字串,如果兩個子字串的相似程度 $\ge 90\%$ 的話,則 Colten 就會將兩筆資料視為是相同的。 

Colten 希望在能夠使得這兩筆資料被視為相同資料的前提下,選擇越長的子字串來比較,請你寫出一隻程式告訴 Colten 他可以使用的最長子字串長度吧

*編輯距離,或稱萊文斯坦距離,其定義為:一個字串經由\textbf{刪除、加入、取代字符串中的任何一個字元}這三種操作得到另一個字串所需的最少操作次數。例如:一個字串 $1112$ 可以透過刪除最後一個字元得到字串 $111$,也可以透過加入一個字元 $1$ 得到字串 $11121$,也可以將最後一個字元取代為 $1$ 得到字串 $1111$,則我們稱字串 $1112$ 與字串 $111$、$11121$、$1111$ 之間的編輯距離為 $1$。

\ProblemInput

輸入第一行有兩個正整數 $n,m$ 代表兩筆資料的長度

輸入第二行有一個長度為 $n$,由 $1\sim7$ 組成的字串 $s_1$ 代表第一筆資料

輸入第二行有一個長度為 $m$,由 $1\sim7$ 組成的字串 $s_2$ 代表第二筆資料


\ProblemOutput

請輸出一個整數代表 Colten 可以使用的最長子字串長度,若無法選擇一個長度 $\ge 10$ 的子字串,則輸出 $-1$

\clearpage

\ProblemConstraints

\begin{itemize}
    \item $10 \le n, m \le 100$
    \item $|s_1|=n$
    \item $|s_2|=m$
\end{itemize}

\ProblemSampleTitle

\begin{ProblemSampleWithNote}{00_01.in}{00_01.out}
在這筆資料中,若我們將整筆資料拿去比對,則獲得的相似程度為 $81.8\%$,不符合 Colten 認定的標準。

但若我們在 $s_1$ 中拿出長度為 $10$ 的子字串 3111111111 或是 1111111112,去和 $s_2$ 中的子字串 2111111111 或是 1111111113,都能夠獲得 $90\%$ 的相似程度,因此答案為 10。
\end{ProblemSampleWithNote}


\ProblemSample{00_02.in}{00_02.out}

\ProblemSubtaskTitle

本題共有 \total{ProblemSubtask} 組子任務,條件限制如下所示。

\begin{ProblemSubtaskTable}
	\ProblemSubtask{19}{$n=m=10$}
	\ProblemSubtask{37}{$10 \le n, m \le 30$}
    \ProblemSubtask{44}{無額外限制}
\end{ProblemSubtaskTable}

\end{document}